% !TEX root = ../TechProject.tex

\graphicspath{{Chapter1/}}

% for example

\chapter{Introduction}
With the rise of streaming services for multiple formats, one interacts with recommendation systems daily. For music streaming, it is an ideal way of filtering the sheer quantity of commercially available music down to an amount that aligns well with one's taste. The prevalence of machine learning has led to this technology being leveraged by streaming services such as Spotify \citep{httpsresearchatspotifycommachine-learning_machine_2023}. For the majority of listeners automating the discovery process has been well-received however, a subset of listeners still finds these systems inadequate.

Recent studies have highlighted that people with objectively "diverse" music tastes usually rely on non-algorithmic ways of finding music \citep{anderson_algorithmic_2020}. This problem is apparent in the continuing popularity of internet radio and services like Bandcamp \citep{market_research_future_internet_2022} \citep{roberts_anti-spotify_2020}. 

DJs are a type of user who falls in the category of having diverse taste. In preparing a DJ set, DJs often source music in non-algorithmic ways \citep{allen_djs_2021}.

The sheer quantity of DJ sets on the internet raises the possibility that this data could be used to train a recommendation system that would provide an effective, algorithmic way to suggest music to this specific audience. This paper presents a recommendation system application trained on DJ sets and compares its recommendation performance with top-performing models.

This paper explores:

\begin{enumerate}
	\item \textit{What are effective methods for the automated recommendation of songs suitable for adding to a given DJ set?}
	\item \textit{Is the proposed method a suitable solution for the automated recommendation of songs suitable for adding to a given DJ set?}
	\item \textit{Can a DJ-set centric dataset be used to solve other DJ tasks?}
\end{enumerate}

Chapters 2 and 3 cover a literature review. Chapter 4 presents gaps in current studies and hypothesises how these gaps could be addressed. Chapter 5 gives an overview of the application and describes example recommendations. Chapter 6 runs through the method of evaluating the application's recommendations. Chapter 7 compares the results found to industry standard models. Finally, chapter 8 concludes the study and discusses further work.
 