% !TEX root = ../TechProject.tex

\graphicspath{{Chapter7/}}

\chapter{Conclusions and Further Work} 

Chapter 1 identified that a dataset of  DJ sets could provide better song recommendations to DJ set-centric genres compared to industry standard models. Research in this field is limited, so a simple recommendation model built around a dataset found on MixesDB was made, and an investigation into its quality was conducted. The following research questions were proposed:

\begin{enumerate}
	\item \textit{What are effective methods for the automated recommendation of songs suitable
		for adding to a given DJ set?}
	\item \textit{Is the proposed method a suitable solution for the automated recommendation of songs suitable for adding to a given DJ set?}
	\item \textit{Can a DJ-set centric dataset be used to solve other DJ tasks?}
\end{enumerate}

Chapter 2 answered Question 1 by establishing how recommendation systems worked and the Diversity issue that baseline music recommendation systems struggle with. Chapter 3 answered Question 3 by analysing machine learning and deep learning advances and how they can benefit the world of DJing. 

Chapter 4 then presents a hypothesis drawn from the knowledge gaps found in the literature review. Chapter 5 presents an overview of how the proposed application works. Chapter 6 then answers the second research question through an experiment. Chapter 7 discusses the results, and this chapter concludes the experiment and discusses how further work can be done.

Section 8.1 will provide a summary of how each of the chapters answered the research questions.
Section 8.2 will highlight the results and how they relate to the proposed research questions. Section 8.3 will discuss the limitations of this study, and further work that could be completed is suggested. Finally, section 8.4 will go over the implications of this investigation.

\section{Findings of each chapter}
This section summarises how each chapter answers the research questions.

Chapter 2 analysed literature in regards to application systems and music taste to answer the following question: 

\textit{What are effective methods for the automated recommendation of songs suitable
	for adding to a given DJ set}

It was concluded that despite recent improvements in music recommendation systems, people with diverse music would often still find music in non-algorithmic ways. A dataset borrowed specifically from listeners with diverse music tastes has yet to be explored fully.

The hypothesis goes on to highlight how this is a problem that affects DJs specifically.
\\

Chapter 3 then examines studies in machine learning to answer the research question:
\\
\\
\textit{Can a DJ-set centric dataset be used to solve other DJ tasks?}
\\
It discusses how advancements in source separation, classification and automatic mixing can benefit the world of DJing. It also identifies a gap in research for recommendation systems explicitly aimed at DJing.

The proposed system was then explained. It is described how the system is built upon the work done by Daniel Chow to create a system that will work for whole DJ sets instead of just a single song. It also explained how it gives a more refined choice due to the use of audio features provided by the Spotify API. Finally, a few examples of inputted DJ sets are provided, briefly describing the findings. 

An experiment was then proposed to attempt to answer the following research question:
\\
\\
\textit{Is the proposed method a suitable solution for the automated recommendation of songs suitable for adding to a given DJ set?}

An evaluation set was made, and the model's best-scoring average R-precision value was found and compared with top-performing models. The proposed model scored significantly lower, concluding that it is not a suitable solution for automating recommendations for a DJ set.

\section{Main Findings of the study}

This section will go over the main discoveries of the paper and how it answers each of the research questions.

\textit{What are effective methods for the automated recommendation of songs suitable
	for adding to a given DJ set}

As stated previously, the most effective method would be to use the Spotify playlist continuer or the SoundCloud method due to the performance of the proposed method not performing well in the given experiment. 

\break
\textit{Is the proposed method a suitable solution for the automated recommendation of songs suitable for adding to a given DJ set?}

This question was answered with an experiment for calculating the model's best R-Precision value, and it was then compared with industry-standard models. The optimal latent factor value was found initially, and then larger DJ sets were used to see if the results improved. Larger DJ sets hindered the r-precision value, and the highest case scored significantly lower than competing models. However, with many areas of the experiment that could be improved, further research should still be done around providing an automated recommendation system, specifically DJ Sets.

\textit{Can a DJ-set centric dataset be used to solve other DJ tasks?}


Despite scoring low, the investigation provides insight into how using a concentrated dataset could better cater to specific audiences. A further developed update on this dataset and model could benefit other areas of DJing. 

This only appears on a handful of mixes on MixesDB, but a time code along with the setlist could provide the foundations for a beneficial DJ tool \citep{mixesdb_2015-01-02_2015}. Hearing instances of an input song getting mixed in and out of could speed up the learning process when they immediately have an audio reference of effective mixing. One hopes further research is done into the large extent of DJ sets available on the internet to hopefully make DJing a more accessible craft. 

\section{Further Work}

A recommendation system that pulls from a dataset of exclusively DJ sets was proposed to see whether it gathers more suitable solutions than the current industry-standard models. 

\subsection{Application}
Despite being an outdated method, using collaborative filtering as the basis of the model felt appropriate due to its continued usage in modern systems. The investigation aimed to test the quality of recommendations pulled from a refined dataset rather than a larger, more general dataset. However, adapting the model to use deep learning with this dataset would have been an excellent opportunity to see whether an improved model would show more desirable recommendations.

A potential implementation was to enable user control of the recommendation process more than just choosing the input songs. It is a believable scenario where a user would appreciate the initial pool of recommendations as much as the refined selection.

The Spotify features play a significant role in refining similar songs to those that have the most similar attributes to the input songs, allowing them to be easily incorporated into a DJ set. This proved helpful in cherry-picking the best songs. 

The vectors used for euclidean distance could have been implemented better, for example, by adding weights to specific attributes, such as prioritising songs at a certain BPM or key. In addition, the weighting for key could have been explored to an interesting length by prioritising the same key but also the relative major/minor or subdominant or dominant keys.

Improvements could have been made to finding an average input vector with the Spotify features. The issue is that one outlier input song could disproportion the input average and spoil the quality of the songs recommended. An example is input songs with vastly different BPMs. Ignoring outlier songs when making an input average could resolve this. A way to still use outlier songs is by comparing them separately and outputting those recommended songs sparingly.

For testing, there were several ways the application could have been assessed further:
\begin{itemize}
	
	\item \textbf{Recommended Song Clicks:} With \textit{Spotify} playlists, 10 songs are recommended based on what is in the playlist. The user can refresh the list, giving out 10 more similar songs. Recommended Songs clicks are the number of refreshes before a missing track shows up.
	
	\item \textbf{Spotify Features:} Testing with Spotify features in conjunction with the Spotify playlist could have been done. One could input 10 songs into both the application and a Spotify playlist. Get the Spotify features of both outputs and compare which ones are the most suitable.
	
	\item \textbf{Spectrogram Analysis} With the dataset being linked to Spotify, a 30-second clip for each song is accessible. Assessing the input and recommended song spectrograms would have given further insight if the songs suggested were appropriate.
	
	\item \textbf{Listening Tests:} Gathering members of the public and getting them to listen to snippets of a given DJ set and comparing the application's and industry-standard suggestions would have given further insight into the subjective quality of the application.
\end{itemize}

\subsection{Dataset}
With MixesDB growing in popularity each year, one hopes to see further exploration in using an expanded version of the dataset to cater to other DJ-related tasks \citep{similarweb_mixesdbcom_2023}. With a sparsity of 99.98\%, it does make sense why this model would underperform in the given experiment \citep{zhang2020alleviating}. An expanded dataset should be used if another investigation in this manner is conducted. 

Despite leading music streaming platforms' huge libraries, many songs, especially ones surrounding DJ sets, cannot be found on these platforms. This is apparent from the many music archive-themed channels on YouTube \citep{allen_djs_2021}. The reason for this is the elusive nature surrounding subcultures that incorporates DJing in a significant way \citep{reynolds_energy_2013}. It is common for some artists to only release content on physical formats like vinyl. This is motivated by a fear of opening up their culture to a majority subset that may risk tainting a pure experience created by a select number of people \citep{wheeler_gentrification_2020}. 

Another reason for the lack of dance music on streaming is the cost of sample clearance \citep{morey_copyright_2013}. More so than other genres (aside from Rap), sampling plays a massive role in dance music. Early instances of DJ-set-orientated genres like House and Techno were built on incorporating drum machines into old disco and soul records \citep{reynolds_energy_2013}. Favourable stylistic features aside, this creates a problem where the shareholders of the given samples are lawfully allowed to sue given artists a significant sum of money for using their given work without permission. This means many DJ set-centric genres, pre the digital age, have had issues being made commercially available. De La Souls 30 year-long quest to regain their highly regarded musical output from the late '80s and '90s is a highly reported instance of this \citep{saunders_soul_2023}.

This inadvertently hinders the proposed recommendation system because it can only input and suggest songs on Spotify. Future work could include songs on YouTube as well as Spotify. YouTube contains a vast amount of songs not on standard streaming platforms. Scraping data from services like Bandcamp would also help to populate the dataset. Combining the software that examines Spotify features on other platforms would be beneficial. However, doing this would require more processing power. A 2020 16GB MacBook Pro used for this investigation would not be sufficient.



\section{Implications}
This section will describe the implications of the data found in this investigation concerning the broader context of music recommendation systems and DJing.

In the context of music recommendation systems, the data found is insignificant, but hopefully, it will start a conversation around using focused datasets to train algorithms aimed at specific audiences. This could draw in audiences with objectively diverse tastes to find music algorithmically.

In the context of DJing, the results also are nothing to behold. However, similar to the comment about music recommendation systems, one hopes that the mass amount of data on the internet surrounding DJ sets will be used to automate tasks surrounding DJing. This has the potential to make DJing more approachable to newcomers.  


% note that \Blindocument has 5 numbered levels, despite setting secnumdepth above. I (and many style guides) would suggest using no more than 3 numbered levels (incl. the chapter), with the option of a fourth unnumbered level.