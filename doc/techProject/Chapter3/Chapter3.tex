% !TEX root = ../TechProject.tex

\graphicspath{{Chapter3/}}

\chapter{Machine Learning used for DJing}

DJing is both a profession and hobby that has been the backbone of many sub cultures since the late 60s. The foundational set up is for one to have two turntables and to transition from one song to another in a stylised or a fluid manner. As the rise of digital audio happened within the 1990s we saw the creation of CDJs and 

\section{Source Separation}

\textbf{explain what it is}

\textbf{general history and music history}

\textbf{Talk about latest system that uses it}

\section{BPM, key, genre classification}

\textbf{explain what it is}

\textbf{general history and music history}

\textbf{Talk about latest system that uses it}

\section{Beat Tracking}

\textbf{explain what it is}

\textbf{general history and music history}

\textbf{Talk about latest system that uses it}

\begin{figure}
\centering
\setlength{\unitlength}{2.4cm}
\begin{picture}(5,4)
\thicklines
\put(1,0.5){\line(2,1){3}}
\put(4,2){\line(-2,1){2}}
\put(2,3){\line(-2,-5){1}}
\put(0.7,0.3){$A$}
\put(4.05,1.9){$B$}
\put(1.7,2.95){$C$}
\put(3.1,2.5){$a$}
\put(1.3,1.7){$b$}
\put(2.5,1.05){$c$}
\put(0.3,3.5){$F=\sqrt{s(s-a)(s-b)(s-c)}$}
\put(3.5,0.4){$\displaystyle
s:=\frac{a+b+c}{2}$}
\end{picture}
\caption[This is also in the LoF]{This is a really really really really really really really really really really really really really really long caption. Well actually, it's not that long; two sentences seems quite reasonable.} 
\label{fig:figure2}
\end{figure}

% note that \Blindocument has 5 numbered levels, despite setting secnumdepth above. I (and many style guides) would suggest using no more than 3 numbered levels (incl. the chapter), with the option of a fourth unnumbered level.