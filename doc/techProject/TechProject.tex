%%% PREAMBLE

\documentclass[11pt,titlepage,oneside]{book}
\usepackage{float}
\usepackage[lmargin=1in,rmargin=1in,tmargin=1in,bmargin=1in]{geometry} % see geometry.pdf on how to lay out the page. There's lots.
\geometry{a4paper}

%%%%%% FOR TESTING ONLY %%%%%%%%%%%%%
\usepackage[english]{babel}
%%%%%%%%%%%%%%%%%%%%%%%%%%%%%%

%%%%%%%%%%%%%%%%% PACKAGES %%%%%%%%%%%%%%%%%

\usepackage{graphicx} % enhanced grpahics support
\renewcommand{\sfdefault}{cmbr} % slightly nicer than the standard sans serif font
\usepackage[onehalfspacing]{setspace} % line spacing
\usepackage[T1]{fontenc} % font encoding (fixes some font-related errors)
\usepackage{textcomp} % more font encoding
\usepackage[printonlyused]{acronym} % To handle acronyms
\usepackage{appendix} % For managing appendices (makes ToC neater)
\usepackage{fancyref}

%%%%%%%%%%%%%%%%% DOCUMENT PROPERTIES %%%%%%%%%%%%%%%%%

% Thesis details
\newcommand{\LongTitle}{Collaborative-Based Filtering Music Reccomendation System for DJ Sets}
\newcommand{\Me}{Harvey James Prewer}

% Number of document levels
\setcounter{secnumdepth}{4}

% Prevent hyphenation
\hyphenpenalty=5000
\tolerance=1000

%%%%%%%%%%%%%%%%% ACRONYMS AND SYMBOLS %%%%%%%%%%%%%%%%%

\newcommand{\RT}{RT\ensuremath{_{\text{60}}}}
\newcommand{\DEG}{\ensuremath{^\circ}}

%%%%%%%%%%%%%%%%% DATE %%%%%%%%%%%%%%%%%

% For placing on title page
\usepackage{datetime}
\newdateformat{MyMonth}{\monthname[\THEMONTH]}
\newdateformat{MyYear}{\THEYEAR}
\newdateformat{MyDate}{\THEYEAR}

%%%%%%%%%%%%%%%%% CAPTIONS %%%%%%%%%%%%%%%%%

\usepackage[margin=2em,singlelinecheck=on,font=sf,labelfont+=bf,labelformat=simple,labelsep=colon]{caption}

\newcommand{\captionfontsize}{\fontsize{9}{11}\selectfont}
\renewcommand\captionfont{\captionfontsize\sffamily}

%%%%%%%%%%%%%%%%% TITLES %%%%%%%%%%%%%%%%%

\usepackage[sf,raggedright,toctitles]{titlesec}

% Chapter heading properties
\titlespacing{name=\chapter}{0ex}{0ex}{.5in}[0ex]
\titlespacing{name=\chapter,numberless}{0ex}{0ex}{.5in}[0ex]

% Subsubsection heading properties
\titleformat{name=\subsubsection,numberless}[hang]{\sf}{}{0pt}{\bfseries}
\titlespacing{name=\subsubsection,numberless}{0em}{2ex}{0ex}

%%%%%%%%%%%%%%%%% TOC / LOF / LOT / LOE %%%%%%%%%%%%%%%%%

\usepackage[titles]{tocloft}

\setcounter{lofdepth}{1}

% put "Chapter #: " in ToC
\renewcommand{\cftchappresnum}{\chaptername~}
\renewcommand{\cftchapaftersnum}{:}
\newlength{\mylen} % a "scratch" length
\settowidth{\mylen}{\bfseries\chaptername\cftchapaftersnum} % extra space
\addtolength{\cftchapnumwidth}{\mylen} % add the extra space

% control indention
\setlength{\cftchapindent}{0em}
\setlength{\cftfigindent}{0em}
\setlength{\cfttabindent}{0em}

\newlength{\tocloftindent}
\setlength{\tocloftindent}{2.3em}

\setlength{\cftsecindent}{1.5em}

%\renewcommand{\@pnumwidth}{1.9em} % was 1.55em
%\renewcommand{\@tocrmarg}{2.9em plus1fil} % raggedright toc entries (no hyphenation)

\setlength{\cftsubsecindent}{3.8em}

%%%%%%%%%%%%%%%%% TABLES %%%%%%%%%%%%%%%%%

\usepackage{array,color,colortbl,multirow,longtable}

\newcommand{\colheading}[1]{\multirow{2}{*}{\textbf{#1}}} % custom column heading command
\newcommand{\tablesubtitle}[2]{\multicolumn{#1}{l}{\emph{#2}}} % custom table sub heading to span the specified number of columns
\renewcommand\arraystretch{1.2} % row height

%%%%%%%%%%%%%%%%% MATHS %%%%%%%%%%%%%%%%%

\usepackage{amsmath,amsbsy,amssymb}

\allowdisplaybreaks

% define new symbols and operators
\newcommand{\R}{\mathfrak{R}} % Real operator
\newcommand{\Z}{\mathbb{Z}} % set of integers
\newcommand{\N}{\mathbb{N}} % set of natural numbers
\newcommand{\F}{\mathcal{F}} % Fourier operator
\newcommand{\ud}{\,\text{d}} % d in differential
\newcommand{\fs}{f\negthinspace{}s} % sampling frequency
\DeclareMathOperator*{\argmax}{arg\,max} % arg min
\DeclareMathOperator*{\argmin}{arg\,min} % arg max
\providecommand{\abs}[1]{\lvert#1\rvert} % abs brackets

%%%%%%%%%%%%%%%%% CITATIONS %%%%%%%%%%%%%%%%%

\usepackage[round]{natbib}

\setcitestyle{aysep={,},notesep={: },citesep={;},yysep={,}}
%\setcitestyle{authoryear,open={(},close={)}}

\setlength{\bibhang}{0pt}

\newcommand{\citepos}[1]{\citeauthor{#1}'s \citeyearpar{#1}}
\newcommand{\citequote}[1]{\newline \strut\hfill \citep{#1}}
\newcommand{\cpage}[1]{p.~#1}
\newcommand{\cpages}[2]{pp.~#1--#2}
\renewcommand\bibname{References}

%%%%%%%%%%%%%%%%% HEADER/FOOTER %%%%%%%%%%%%%%%%%

\usepackage{fancyhdr}
\setlength{\headheight}{15pt}

\newcommand{\myfooter}{\textsf{\thepage}}

\fancypagestyle{plain}{%
	\fancyhf{}%
	\fancyfoot[R]{\myfooter}%
	\renewcommand{\headrulewidth}{0pt}%
	\renewcommand{\footrulewidth}{1pt}%
}

\pagestyle{fancy}
\fancyhf{}%
\renewcommand{\headrulewidth}{1pt} %header line width
\renewcommand{\footrulewidth}{1pt} % footer rule width
\fancyhead[R]{\textsf{\nouppercase{\leftmark}}} % header right - chapter title
\fancyfoot[R]{\myfooter} % footer right - page number

%%%%%%%%%%%%%%%%% HYPERREF %%%%%%%%%%%%%%%%%

% this should hopefully be the last entry in the preamble - defines properties of the typeset pdf file
% this package makes clickable links in the pdf
\usepackage[colorlinks,breaklinks,pdfdisplaydoctitle,plainpages=false]{hyperref} 
\hypersetup{pdftitle={\LongTitle},pdfauthor={\Me},
	citecolor=black,%
	filecolor=black,%
	linkcolor=black,%
	urlcolor=black
}

\renewcommand{\equationautorefname}{Equation}
\renewcommand{\figureautorefname}{Figure}
\renewcommand{\Itemautorefname}{Item}
\renewcommand{\tableautorefname}{Table}
\renewcommand{\sectionautorefname}{Section}
\renewcommand{\subsectionautorefname}{Section}
\renewcommand{\subsubsectionautorefname}{Section}
\renewcommand{\chapterautorefname}{Chapter}
\renewcommand{\partautorefname}{Part}

%%%%%%%%%%%%%%%%%%%%%%%%%%%%%%%%%%%%%%%%%%%%
%%%%%%%%%%%%%%%%% DOCUMENT %%%%%%%%%%%%%%%%%
%%%%%%%%%%%%%%%%%%%%%%%%%%%%%%%%%%%%%%%%%%%%

\begin{document}
	
	\pdfbookmark[0]{Title page}{title} % Sets a PDF bookmark for the title page
	\setlength{\parindent}{0pt} % set paragraph indent
	\setlength{\parskip}{0ex} % set paragraph spacing
	\pagenumbering{alph} % prevents warnings about duplicated page numbering
	
	% !TEX root = ../TechProject.tex

\graphicspath{{FrontMatter/}}

%%%%%%%%%%%%%%%%% TITLE PAGE %%%%%%%%%%%%%%%%%
\begin{titlepage}
\thispagestyle{empty}
\hfill\includegraphics[width=50mm]{images/UniS_Logo.pdf}
{\sf
\centering
\null\vfil\vfil
	{
		\huge\LongTitle\\
	}
\vfil\vfil\vfil
	{
		\Large\Me\\
	}
\vfil\vfil\vfil
	{
		\Large{}BMus/BSc Music \& Sound Recording (Tonmeister)\\
		Technical Project\\
	}
\vfil\vfil
	{
		\Large{}Institute of Sound Recording\\
		University of Surrey\\
	}
\vfil
	{
		\Large\MyMonth\today~\MyYear\today\\}
	}

	{
		\includegraphics[width=40mm]{images/University_of_Surrey_coat_of_arms.pdf}
		\vspace{-.4in}
	}
\end{titlepage}

\setlength{\parskip}{1ex plus 0.2ex minus 0.2ex} % set paragraph spacing
\onehalfspacing

%%%%%%%%%%%%%%%%% ABSTRACT %%%%%%%%%%%%%%%%%

\pdfbookmark[0]{Abstract}{abstract}

\chapter*{Abstract}
\thispagestyle{empty}
\setcounter{page}{2}

This is my abstract.

%%%%%%%%%%%%%%%%% ACTUAL FRONT MATTER %%%%%%%%%%%%%%%%%
\frontmatter

\singlespacing

\setlength{\parskip}{0ex} % set paragraph spacing

\clearpage
\pdfbookmark[0]{Contents}{Contents}
\phantomsection
\tableofcontents
\clearpage
\phantomsection
\addcontentsline{toc}{chapter}{List of Figures}
\listoffigures
\clearpage
\phantomsection
\addcontentsline{toc}{chapter}{List of Tables}
\listoftables

\setlength{\parskip}{1ex plus 0.2ex minus 0.2ex} % set paragraph spacing

\onehalfspacing

\chapter{Acknowledgements}

Thanks to Christos for the encouragement and advice whilst doing this.
	
	%%%%%%%%%%%%%%%%% MAIN %%%%%%%%%%%%%%%%%
	\mainmatter
	
	% change header
	\renewcommand{\chaptermark}[1]{\markboth{\chaptertitlename\ \thechapter: #1}{}}
	
	\pagenumbering{arabic}
	\setlength\abovedisplayshortskip{0.5\baselineskip}
	\setlength\belowdisplayshortskip{0.5\baselineskip}
	\setlength\abovedisplayskip{0.5\baselineskip}
	\setlength\belowdisplayskip{0.5\baselineskip}
	
	%%%%%%%%%%%%%%%%%%%%%
	%%%%%%%%%%%%%%%%%%%%%
	% For example
	
	%\include{Introduction}
	
	% The argument for \include is a filename of a LaTeX file contained within the same folder at this file. It could also be placed in a subfolder; the argument can then be specified as {subfolder_name/filename}.
	
	% The \include'd file contains LaTeX commands that will, more or less, be read as if the code was contained in this file at the point the \include was called.
	
	% There is an equivalent command, \includeonly, which you should place at the top of your document. Use this command to choose the files to include (rather than commenting lines). In this way, the document will be typeset as if the files were present (so links and cross-references won't be broken), but any output associated with the files will be excluded.
	
	% Another tip: If using TeXShop, which uses synctex, you can command click on the preview to find the corresponding commands in your source. The reverse operation is also permitted. To do this from \include'd files, you need to tell TeXShop what the root file is. This is done by placing this line AT THE TOP of each \include'd file (you must include the "%"):
	% ! TEX root = root_filename.tex
	% Doing this has other advantages, like being able to typeset from an included file, rather than having to typeset from the root file.

% Chapter 1
% !TEX root = ../TechProject.tex

\graphicspath{{Chapter1/}}

% for example

\chapter{Introduction}
With the rise of streaming services for multiple formats, one interacts with recommendation systems daily. For music streaming, it is an ideal way of filtering the sheer quantity of commercially available music down to an amount that aligns well with one's taste. The prevalence of machine learning has led to this technology being leveraged by streaming services such as Spotify \citep{httpsresearchatspotifycommachine-learning_machine_2023}. For the majority of listeners automating the discovery process has been well-received however, a subset of listeners still finds these systems inadequate.

Recent studies have highlighted that people with objectively "diverse" music tastes usually rely on non-algorithmic ways of finding music \citep{anderson_algorithmic_2020}. This problem is apparent in the continuing popularity of internet radio and services like Bandcamp \citep{market_research_future_internet_2022} \citep{roberts_anti-spotify_2020}. 

DJs are a type of user who falls in the category of having diverse taste. In preparing a DJ set, DJs often source music in non-algorithmic ways \citep{allen_djs_2021}.

The sheer quantity of DJ sets on the internet raises the possibility that this data could be used to train a recommendation system that would provide an effective, algorithmic way to suggest music to this specific audience. This paper presents a recommendation system application trained on DJ sets and compares its recommendation performance with top-performing models.

This paper explores:

\begin{enumerate}
	\item \textit{What are effective methods for the automated recommendation of songs suitable for adding to a given DJ set?}
	\item \textit{Is the proposed method a suitable solution for the automated recommendation of songs suitable for adding to a given DJ set?}
	\item \textit{Can a DJ-set centric dataset be used to solve other DJ tasks?}
\end{enumerate}

Chapters 2 and 3 cover a literature review. Chapter 4 presents gaps in current studies and hypothesises how these gaps could be addressed. Chapter 5 gives an overview of the application and describes example recommendations. Chapter 6 runs through the method of evaluating the application's recommendations. Chapter 7 compares the results found to industry standard models. Finally, chapter 8 concludes the study and discusses further work.
 

% Chapter 2
% !TEX root = ../TechProject.tex

\graphicspath{{Chapter2/}}

\chapter{Machine Learning used in Music Recommendation Systems}

In this chapter Ill be answering the research question:

\textit{What are effective methods for the automated recommendation of songs suitable for adding to a given DJ set} 

\section{The need for music recommendation systems}
Spotify, SoundCloud, Apple Music and other streaming services gives one access to a library of tens of millions songs. Music Recommendation systems are excellent at fitting a users preference whilst incorporating some level of filtering of an overwhelming amount of songs \citep{bollen_understanding_2010}. 

Music recommendation systems study users habits and taste, and with that information give out suitable recommendations. This aids the listener to discover new songs or artists they otherwise would not have found.

With this potential, a well made recommendation system could be a make or break for choosing one stream service from another. For this reason a lot of research gets put into recommendation systems as an attempt to retain engagement. 

 With recommending music, a lot of sub conscious factors come into play on what a user wants to listen to at a given time. This can range from characteristics and mood of the listener \citep{ferwerda_personality_2015}  \citep{rentfrow_re_2003}, to what they get up to in day-to-day life \citep{gillhofer_iron_2015} \citep{wang_context-aware_2012}.  The users environment can have an affect as well \citep{kaminskas_location-aware_2013}. Observing a user made playlists also can reveal a lot on what groups of songs work for suited situations \citep{zheleva_statistical_2010} \citep{mcfee_hypergraph_2012}.
 
 A necessity for making a recommendation system suited for a type of product is taking into consideration the common place attributes of the product. Music lends itself to having a specific method due to short duration and high emotional connection. A recommendation system that uses these attributes to its advantage will be more successful than ones that don't.

\section{Collaborative Filtering}

Collaborative Filtering guesses users taste by looking at similar user-item connections. Using an explicit example, if a user rates something highly, it can provide similar suggestions from looking at other users ratings \citep{celma_recommendation_2010}.

\begin{figure}[H]
	\includegraphics[scale=0.65]{images/collaborative_filtering}
	\centering
	\caption{User-item matrix used for collaborative filtering \citep{celma_recommendation_2010}} 
	\label{fig:figure}
\end{figure}

Collaborative filtering works by taking a matrix of users and items where some form of interaction is measured. Examples of interaction's could be plays of a song, rating of a movie/product or screen time on an app. In figure 2.1, i represents items, and u is users.

The first known use of collaborative filtering is with Goldberg's Tapestry system, a mailing list filter where users collectively decide which type of emails get the most importance \citep{goldberg_using_1992}. The first instance of collaborative filtering being used was with a music based system called \textit{Ringo} where users would enter in music they rated and would get recommendation's pulled from similar users \citep{shardanand_social_1995}. 

Despite evaluations in deep learning and neural networks. Collaborative Filtering is still used today in the state of art reccomendation systems. The winner of recsys playlist continuer challenger in 2018 used a combination of collaborative filtering and deep learning in there submitted model \citep{volkovs_two-stage_2018}.

Collaborative filtering can be divided into three types:




\subsection{Explicit Feedback}

\textit{Ringo} and many of the first music recommendation systems that use collaborative filtering uses explicit feedback for its data. Explicit feedback is when the system asks a user for some form of measurement about the likeability of said product. \citep{celma_recommendation_2010}

An example of explicit feedback being used is in the RACOFI (Rule-Applying Collaborative Filtering) system. Ratings are used to find recommendations through collaborative filtering and then the application apply logic rules to decipher the most suitable recommendations further.\citep{anderson_racofi_2003}. Its added logic rules implicitly changes the users previous ratings. Other reccomendation systems start having similar rules including \textit{Indiscover} and Slope One \citep{celma_music_2010} \citep{lemire_slope_2007}.

\subsection{Implicit Feedback}

Recommendation systems that take only implicit feedback will focus on what the listener interacts with, rather than asking it for its opinion on the said content. The main reason implicit feedback is frowned upon is using this method doesn't give a scale of enjoyability of the content, in the context of music, just whether the user listened to said song or artist how many times. An example of this with Spotify is if another person uses there account or if they left a device on autoplay as well on mute. However, collecting implicit data is a lot easier because it just requires engagement with the said application to get valuable information. 

There is also a lot to be found with implicit data. In 2018, a music recommendation system took the time in the day which users listened to music and gained successful results \citep{sanchez-moreno_incorporating_2018}. Takama et al took this a step further including time as well as data, nationality and content features found through spotify \citep{takama_context-aware_2021} 

The two different approaches used for data finding with Collaborative Filtering is Item-based Neighbourhood or User-based neighbourhood. 

\subsection{Item Based}

\textbf{Talk about the definitions and examples of these things \citep{ricci_recommender_2011}}

Item-Based Neighbourhood is when similarites for a given item is found based on the users previous item ratings. Below shows a matrix with ratings from  $u_{2}, u_{i}$ and $u_{m-1}$. For finding similarities between $i_{i}$ and $i_{k}$, we only take $u_{2}$ and  $u_{i}$ into consideration when using item based neighbourhood because they have rated both item $i_{i}$ and $i_{k}$.  

\begin{figure}[H]
	\includegraphics[scale=0.65]{images/neigbourhood_based}
	\centering
	\caption{Matrix showing given ratings. \citep{celma_recommendation_2010}} 
\end{figure}

There are many ways to calculate how similar two items, examples being cosine similarity, Pearson correlation or adjusted cosine,. For Cosine similarity, the inner product space is taken into account for calculating its similarity, it can be described with the following equation.

\begin{equation}
		sim(i , j) = cos( \textbf{i}, \textbf{j} ) = \frac{ \textbf{ i }, \textbf{ j }}{ || i || * || j || } = \frac{ \sum_{ u \in U } r_{ u, i }, r_{ u, j }} { \sqrt{ \sum _{  u \in U } r^{2}_{ u , i}} \sqrt{ \sum _{  u \in U } r^{2}_{ u , j}}}
\end{equation}

Cosine similarity is not advised in this scenario because users usually have there own personal rangers when it comes to rating. Adjusted cosine similarity is good because it makes use of the users average rating when deciphering similarities, equation is shown below.

\begin{equation}
	sim(i , j) = \frac{ \sum_{ u \in U } ( r _{ u, i } - \bar{r} _{u} ) ( r _{ u, j} - \bar{r} _{u} ) } { \sqrt{\sum_{ u \in U } ( r _{ u, i } - \bar{r} _{u} )^2} \sqrt{\sum_{ u \in U } ( r _{ u, j } - \bar{r} _{u} )^2}}
\end{equation}

Pearson correlation gives a coefficient value from 1 to -1, 1 showing strong correlation with a positive gradient, and -1 display effective correlation with a negative change.

\begin{equation}
	sim(i , j) = \frac{ Cov( i, j) }{ \sigma _{i} \sigma _{j} } = \frac{ \sum _{ u \in U} ( r _{ u, i } - \bar{r} _{u} ) ( r _{ u, j} - \bar{r} _{u} ) } { \sqrt{\sum_{ u \in U } ( r _{ u, i } - \bar{r} _{u} )^2} \sqrt{\sum_{ u \in U } ( r _{ u, j } - \bar{r} _{u} )^2}}
\end{equation}

After the similarities are calculated, the next step is to predict how the user would rate the item in question. A way of doing this is calculated a weighted sum of the users previous item ratings. Allowing $ S^{k}(i;u)$ to equal the list of items i user u has rated, equation below shows how the predicted value is found.

\begin{equation}
	\hat{r} _{u,i} = \frac{ \sum _{ j \in S^{k}(i;u)} sim(i , j) r _{u, j}}{\sum _{j \in S^{k}(i;u)} sim(i , j)}
\end{equation}




\subsection{User Based}

User-Based Neighbourhood is when you look users who rate similar to see whether item i is similar to user u. Below shows an equation similar to the one above but instead we look through a list similar users.

\begin{equation}
	\hat{r} _{u,i} = \bar{r}_{u} + \frac{ \sum _{v \in S(u)^{k}} sim(u ,v) ( r_{v, i} - \bar{r}_{v})}{\sum _{v \in S(u)^{k}} sim(u , v)}
\end{equation}

For one to calculate $sim(u , v)$, you'd use Pearson Correlation, Cosine similarity or matrix factorisation.

Item based is when the system predict rating of user u for item i rooted from ratings of u for similar items to i. Similarity between two items are affirmed when multiple users rate them alike.

\subsection{Matrix Factorisation}

Matrix factorisation is when you take a sparse matrix and instead of storing each rating for each user, you store features that when multiplied can calculate the rating of said item by said user.

This is really useful for a sparse matrix, which is usually the case for most reccomendation systems, because it shows information that the matrix alone doesnt. Because it factorises the users ratings from features, you can see trends on what groups of users likes and doesnt like. Because it reducing the dimensionality of a given matrix, matrix factorisation requires less processing power than other methods of finding similar items. 

There are different ways of factorising a matrix. An example of one is Singular Value Decomposition (SVD). Heres the equation with U and V being the number of matrices for a given amount dimensions:

\begin{equation}
	M = U \sum V ^{T}
\end{equation}

Theres other means to compute M. Least Squares, uses a computational safe way to make sure M doesnt go to off from its original value, whilst stochastic gradient repetitively uses random bits of data to approximate U and V \citep{koren_matrix_2009}. 

When the matrix is split up, the predicted rating can be calculated from the user and item feature vectors. 

\begin{equation}
	\hat{r} _{u,i} = U _{u} . V _{i}^{T} = \sum_{f=0}^{k} U_{u,f} V_{f, i}
\end{equation}

Looking at the latent factors one can find  similar items using a cosine similarity.

\subsection{Limitations}
Despite being popular, there are a number of reasons why wouldn't use Collaborative filtering.

\textbf{Sparse Data - }Not every user has listened to every song, far from it. This means in data sets, having a sparsity of around 98-99 \% is very common.

\textbf{Grey sheep - }This is when a user has a unique taste not similar to a lot of other users, making it hard to stem recommendation's from. This a common problem with datasets that are very sparse \citep{claypool_combining_1999}.

\textbf{Cold Start - } When there is new users or items, theres little data associated with them so it makes it challanging to come up with suitable reccomendations based on either the said user or item. The term cold start refers to new items and the term refered to new users is called early raters \citep{avery_recommender_1997}.

\textbf{Popularity Bias - } Another problem with Collaborative Filtering doesnt take into account any information about the item, only users interactions with it. This means that it has a bias towards popular items.

\textbf{Feedback loops - } When users interact with items that are recommended through CF, based on previous user item interactions, it strengthens the initial recommendations more and creates a loop \citep{sanchez-moreno_incorporating_2018}.


For neighbourhood-based, recommendation can either be user-based or item-based. The Ringo system is a good example of user based recommendation's, where it assesses the taste of a user for an item using the ratings of the item from different users. These users with similar rating trends are called neighbours. 

\textbf{Talk about latest music reccomendation system that uses it (recsys one)}
\section{Content Based Filtering}

Content Based Filtering works by looking at the characteristics of a given item and see if it matches the preference of the user. Recommendations do not stem from what other users interact with, it only stems from the given information about the item. Its based on finding information about the item and filtering \citep{casey_content-based_2008}, and within the context of music, it recommends songs that has similar information to what the user already enjoys \citep{aucouturier_music_2002} \citep{logan_music_2004}. Theres been a lot of research on extracting the componenets of an audio file of music \citep{ribecky_multi-input_2021} \citep{zhao_musical_2022}. Recent developments have shown successful extractions of genre and structure with the MusicBERT Model \citep{zhu_musicbert_2021}. It's found that midi extraction from songs can aid to give better solutions for less popular songs, helping eliminate the cold start problem \citep{yadav_improved_2022}.

Early uses of content based filtering were text based, because of its ease of information extraction, An example of this is the PRES system \citep{van_meteren_using_2000}. But now with advances in machine learning, more complex formats can be used for extraction like images or audio. Models like "Xception" have proven successful at extracting audio features from songs \citep{chollet_xception_2017} \citep{singh_robustness_2022}.

When finding similar items with CB, it simply looks at how similar the attributes of each items is, without introducing any subjective factor, like user behaviour, into its decision making. Thinking of an item being a vector made of its defined values of attributes, we can find the distance between each item. Its common to have these values be numerical. Common ways of caluclating the distance is Euclidean, Manhattan, Chebychev, cosine distance for vectors, and Mahalanobis distance.

\begin{equation}
	d(x,y) = \sqrt{\sum _{i=1} ^{n}(x_{i} - y_{i})^{2}}
\end{equation}

\begin{equation}
	d(x,y) = \sum _{i=1} ^{n} | x_{i} - y_{i} |
\end{equation}

\begin{equation}
	d(x,y) = man_{i} = _{1 . . n} | x_{i} - y_{i} |
\end{equation}

\begin{equation}
	d(x,y) = \sqrt{ ( x - y )^{ T } S^{ -1 } ( x - y ) }
\end{equation}

Euclidean, Manhattan and Chebychev distance are used when there is little relationship or correlation between attributes. If there is correlation, its advised to use Mahalanobis distance \citep{celma_recommendation_2010}.

When the attributes aren't measured numericlly, one uses a delta function. When two attributes match it equals zero, other wise it equals 1.

\begin{equation}
	d(x,y) = \omega \sum _{ i = 1 } ^{ n } \delta (x _{i}, y _{i})
\end{equation}

Content Based Filtering overcomes the problems CF has of being able to rate items that previously havent had any ratings. Its also able to adapt to any changes in the users preference quickly \citep{isinkaye_recommendation_2015}. For users who dont want to share there data they can get suitable reccomendations as well \citep{k_you_2006}.

Another methods of comparing includes Clustering. Clustering is when one  groups a collection of data objects, so that some objects clustered together are very similar, while other clustered objects are not alike. Similarity is then evaluated based on chosen parameters \citep{ferretti_clustering_2018}. A popular algorithm is the k-mean cluster where each cluster is representerd by a mean value of the object \citep{han_data_2006}. A recent system that uses clustering uses a altered version to k clustering to limit the randomness in suggestion you usually get with this method \citep{chang_personalized_2017}.



\subsection{Limitations}
The cold start and grey sheep problem also occur in content based filtering. It's often the case where popular items would have better defined features, and older users well have better represented features.

\textbf{Novelty Problem - } This is when the user is recommended items too similar to the one there profile, an application would need some way of diversifying recommendations to overcome this.

\textbf{Retrieving metadata - } Even though we have seen recent improvements in extracting meaningful meta data \citep{vall_feature-combination_2019} \citep{singh_novel_2022}. One uses this data in deep neural networks or hybrid models. We aren't yet at a point where we can obtain rich enough meta data to rely on Content Based Filtering alone.

\textbf{Suggestions not opinionated- } It doesn't take into the account the opinions of users, only description of what the item is. This may lead to some poor quality recommendations.  


\section{Diversity issue}

Recommendation systems are used extensively by majority of spotify users. However, there are a subsection of users who, despite huge breakthroughs dont rely heavily on algorithmic means of discovering music. In 2020, there was a study done that found users with more diverse tastes would find music in non algoritmic ways \citep{anderson_algorithmic_2020}. This study also found state of the art reccomendation systems performed better with users with less diverse taste. and users who would gain a more diverse taste over time, would do so by exploring non algorithmic ways of finding music.

A reason for this is because alogorithmic ways go above and beyond giving personalised suggestions, one can suffer from choice overload when using algorithmic ways to find music \citep{iyengar_rethinking_1999}. This may do with the lack of refinement when it comes to training models, lot of MRS dont focus on a specific type of person when training a model \citep{laplante_improving_2014}.

A reason for this issue could be the way classification is handled in music reccomendation systems, a debate that is passed from generations \citep{moles_sociodynamique_2019} \citep{dimaggio_classification_1987} \citep{bourdieu_distinction_2010}. In the context of a reccomndation system, a genre is usually just treated as a "tag" and its social and cultural relevance is often pushed to the side when treated in such a way \citep{porcaro_diversity_2021}. The act of a machine understanding a intensely subjective concept as genre is one that is proven challenging \citep{nurnberger_survey_2014}. Vlegels attempts to overcome this by organise groupings based on user artist relationship, rather than artists genre \citep{vlegels_music_2017}.

This problem itself has saw another rise in the market of "curation". Having professional spot interesting things and document them to consumers. With this comes the role in the music world of a DJ, who way before recommendation systems existed were doing just that. Recently we have seen attempts of integrating DJs somehow in the algorithmic world. This is partly motivated by how much a company can pedel a term like "curation" in the music world and see it as a necessity \citep{barna_perfect_2017}. One can see this in the rise of internet radio as well.

Internet Radio is a strangely thriving industry which is expected to have an net worth of 9.2\$ billion \citep{market_research_future_internet_2022}. There are many different factors on why this resurgence has occurred on a platform that one could easily see die. The quoted figure would also include radio beyond the scope of music. But specifically in dance music centric radio has also received a huge rise \citep{gillett_how_2021}. It includes increased speeds in mobile data, and the choice of stations, but one can see the push for curation from leading companies in this field \citep{deane_media_apple_2020} \citep{nts_radio_nts_2023}.

The closest integration we got with DJ's and the algorithmic centric streaming apps is Apple Musics Beats 1 Radio station where apart of the subscription service you got access to a radio station of the most popular artist and DJs in the western world broadcasting shows, providing entertaining commentary and selected musical recommendations. Another is spotifys curated playlists, and more specifically there track ID's series, where they get established DJs to make a huge ever changing playlist of songs they play out live. The only instance I found of combining the two is a medium article written by Daniel Chow where he made a recommendation system out of the many DJ mixes found on mixesDB which takes in a single song and looking through the dataset find similar songs \citep{chow_music_2020}. Within the context of a DJ set finding recommendationd from multple songs, or a whole other set is something that has been previously unexplored, and could provide with solutions with assuring a exciting level of diversity that is missed in many recommendation systems.

% note that \Blindocument has 5 numbered levels, despite setting secnumdepth above. I (and many style guides) would suggest using no more than 3 numbered levels (incl. the chapter), with the option of a fourth unnumbered level.

% Chapter 3
% !TEX root = ../TechProject.tex

\graphicspath{{Chapter3/}}

\chapter{Machine Learning used for DJing}

\begin{figure}[H]
	\includegraphics[scale=0.3]{images/pioneers_history}
	\centering
	\caption{Pioneer CDJ models from 1992-2012 \citep{chesters_history_2017}} 
\end{figure}

This chapter answers the following research question:
\\

\textit{Is the proposed method a suitable solution for the automated recommendation of songs suitable for adding to a given DJ set?} 
\\
\\
DJing is a practice that has been the backbone of many sub cultures since the late 60s \citep{brewster_last_2014}. DJing has contributed to the evolution of various genres like Disco, Reggae, and the many forms of electronic dance music \citep{partridge_dub_2010} \citep{reynolds_energy_2013}. DJing traditionally involves two turntables and to transition from one song to another in a stylised or fluid manner. As the rise of digital audio happened within the 1990s saw the creation of CDJs.. CDJs introduced mixing tool softwares such as Pioneers rekordbox. Mixing tool softwares makes use of music information retrival techniques to lower the difficulty of organising and preparing for a DJ set \citep{kim_automatic_2017}. Different implementation of music information retrieval varies from source separation and beat tracking. These tasks involve machine learning and deep learning algorithims, and are now fully embedded in DJ softwares. 

\section{Source Separation}

Source separation is the process of estimating specific sources in one mixed signal. In the context of music, its often used to separate an instrument or voice within a track \citep{sgouros_efficient_2022}. With the recent feats in Deep Learning there has been mass advances in source separation in different fields. One noticable one is the use of U-nets, a convolutional neural network that proved very effective in segmenting biomedical images \citep{ronneberger_u-net_2015}. With the use of spectrograms, a very similar got used in a music source seperation model \citep{jansson_singing_2017}. The following model is then further adapted on the highly popular open source separate made by Deezer called Spleeter \citep{hennequin_spleeter_2020}. Despite no official methods been published, popular DJ manufactor released the Serato Stems update to there DJ software which provides very similar functionality to spleeter but in real time \citep{kirn_review_2023}. This has opened up the possibilities for a DJ to make remixes in real time, separating certain instruments or voices from certain songs.


\section{BPM, key, genre classification}
Briefly touched upon in context based filtering, Audio classification is the building of metadata surrounding a piece of audio by analyzing it \citep{sharma_audio_2021}. As well as proving useful for finding similar songs, audio classification models is used extensively in DJ softwares for calculating tempo, key and other musical attributes.

For key detection, standard machine learning techniques proves powerful enough. The Support Vector Machine proposed by George et. al had an accuracy value of 91.49\% and performed better compared to previous papers \citep{george_development_2022}. Having key information is essential within DJing because assuring the keys either match or modulate functionally assures fluidity in the transition. 

With tempo detection, the current state of the art makes use of Temporal Convolutional Networks to estimate the tempo, as well the up and down beat \citep{bock_deconstruct_2020}. The foundation of the model was explored previously \citep{bock_multi-task_2019}, but found that incorporating an extra dilation rate to each layer of the model gave more accurate results. Knowing the tempo of a given song is one of the main draws of CDJs over vinyl turntables, so further developments in tempo detection will inevitably find itself implemented in DJ software's.

As the case for the BPM and key, the advancement of genre classification has had some significant contirbutions in recent years. The most recent signification is a model that uses short-time fourier transform, pitch, timbre and NMF feautres are extracted from a given piece of audio, which is then fed through a deep belief neural network and a Wale Integreated SnO algorithm \citep{kumaraswamy_optimal_2022}. Advancements in genre classification would be helpful within the DJ world as it would help the organisation of playlists for a given DJ.

\section{Automated Mixing}
As well the advances of classification and separation that will aid a DJ, there are an equal amount of advances that could replace them. In February 2023, Spotify began unravelling its brand new DJ feature, combining situational music recommending accompanied by an AI generated host, mimicking the role of a personal broadcast DJ \citep{naomi_spotify_2023}. Within the research world, there have been advances on not just the mimiking of a human DJ's dialogue, but also the way in which a human would transition from one song to the next.

Spotify have some level of implementation to this already and have ran experiments to show a general further appreciation to playlist if it includes DJ-style transitions \citep{bittner_automatic_2017}. Bittners playlist sequencer also included an attempt of DJ-style tranistions, they used peak detection to determine where abouts in the song the "drops" occurs, and gave specific rules to assure appropriate drop in and drop out points for a given track \citep{bittner_automatic_2017}.  However there means of working out up and down beat was ineffective and greatly effected the fluidity of the transitions. A recent attempt at just the transition aspect of this found that using Boykov-Kolmogorov algorithm alongside spectrogram analysis of the two given songs made for great results for songs with similar tempos and keys \citep{robinson_automated_2023}.

\section{Summary}
\textbf{\textit{Write a summary}}

% note that \Blindocument has 5 numbered levels, despite setting secnumdepth above. I (and many style guides) would suggest using no more than 3 numbered levels (incl. the chapter), with the option of a fourth unnumbered level.

% Chapter 4
% !TEX root = ../TechProject.tex

\graphicspath{{Chapter4/}}

\chapter{Hypothesis}

\section{Literature Review Conclusions}

The literature review set out to answer the two questions:
 
 \begin{enumerate}
 	\item \textit{What are effective methods for the automated recommendation of songs suitable for adding to a given DJ set?}
 	
 	\item \textit{Can the proposed method be further developed to solve other DJ tasks?}
	\end{enumerate}
 
 It was shown that the current machine learning advances in separation and classification could aid a DJ greatly, but despite great advances in recommendation systems, algorithmic ways of gathering song recommendations do not handle the cultural significances of genre and style.

With this being the case, modern day recommendation systems don't usually cater to the traits found in both professional and hobbiest DJs. Preparing a DJ set requires a sizeable amount of music \citep{allen_djs_2021}. Based on the research ran by Anderson, these group of people would certainly fall in the category of diverse \citep{anderson_algorithmic_2020}.

The necessity to reach a certain quantity of songs to DJ, makes for a laborious passed time of collecting or "digging" for songs. Digging is a pre digital era term, to go find music in specific record stores or from esoteric places, with the higher likelihood of picking up something that could be emotionally resonant \citep{allen_djs_2021}.

Recommendation systems could be seen as a replacement for digging. A given application should knows if a user want esoteric music. But as mentioned in the Diversity Issue sub-section, finding these sorts of recommendations relies heavily on the cultural context surrounding a genre and with it gives less desirable suggestions compared to manually finding music. 

This problem is somewhat resolved with the popularity behind sites like "bandcamp", that prides itself with regularly updated blog posts, and the option to examine other people collections. Allowing the user to openly "dig" more so than other music based platforms \citep{bandcamp_about_2023}.  These sites can be a great resource to find music in a digging manner, but having an algorithm do it would be more time efficient. 

Successfully training a model from this type of dataset could then be used to assist other DJ related tasks, an example of parameter values was given.

Aside from Daniel Chow's model, a recommendation system that attempts to mimic suggestions found from digging is yet to happen \citep{chow_music_2020}. However, ways of adapting modern day recommendation systems can mimic the effect.

\section{Algorithmic ways of obtaining recommendations from a DJ set}
\begin{enumerate}
	\item \textbf{Spotify }- Not many can be found, but labels like !K7 publish DJ sets commercially \citep{k7_about_2023}.  Find a DJ set on Spotify, add that to a playlist, and look for the"
	Recommended based on the playlist" . The main problem with this is that DJ sets on Spotify are few and far between. One can only input a mix one’s heard from the radio or a recorded live performance by finding the tracklist and adding the songs into a playlist. The system also finds what other users are listening to, often giving song recommendations that are usually quite popular and well-known. Spotify has a vast library of music, but a sizeable chunk of music played on Spotify are not found on DJ sets.
	
	\item \textbf{SoundCloud }- Many DJ sets are found here, and using the "radio" function, recommendations can be found. However, it prioritises other DJ sets on its "radio " when getting recommended songs is more desirable in this context. The recommendation system is also only limited to what is available on SoundCloud.
	
\end{enumerate}

\section{Hypothesis proposal}
	
The hypothesis is that the application will give song recommendations that are both stylistically suitable but aren’t necessarily popular. Its common for a DJ, having a diverse taste, to go into
great lengths to find obscure and unknown songs. Therefore, not prioritising popularity is a
desirable trait in a recommendation system for a DJ’s music consumption and knowledge, usually
being more than the average consumer. Its a common trait in most music recommendation
systems to prioritise popularity. Building a recommendation system from a dataset of DJ sets
will hopefully give these types of suggestions.	

With DJ sets being the intended input of this model, one hopes that its accuracy is comparable with top performing recommendation system models.

\section{Summary}
The literature review was summarised and the gaps in research was stated. An overview of obtaining recommendations for a DJ set was given. A hypothesis given  hopes the accuracy of the model would be comparable to top performing model, when DJ sets are used as inputs. 



% note that \Blindocument has 5 numbered levels, despite setting secnumdepth above. I (and many style guides) would suggest using no more than 3 numbered levels (incl. the chapter), with the option of a fourth unnumbered level.

% Chapter 5
% !TEX root = ../TechProject.tex

\graphicspath{{Chapter5/}}



\chapter{Application}

Inspired by the training data from a fader estimation investigation, \textit{MixesDB} is a website of archived DJ sets \citep{kim_automatic_2017}. With over 260,000 mixes on the site, using this as a dataset to train the model proved to be an excellent choice \citep{mixesdb_main_2023}. As mentioned in Chapter 2, Daniel Chow built a system that takes in a single song and outputs similar songs based on the \textit{MixesDB} dataset \citep{chow_music_2020}. His code will provide the foundation, which is built upon by the inclusion of multiple inputs and an extra layer of filtering.

\section{Data set}
As Chow details, data was scraped from the website \textit{MixesDB} done using Selenium \citep{chow_music_2020}. The tracks in a given DJ Set which had a corresponding Spotify link were stored with a given user ID for the DJ.  In cases where a single DJ set included multiple DJs, the set was considered as belonging to all the DJs involved, and the corresponding songs were linked to each DJ's user ID.  With the article being dated back in 2020, the data set only has DJ sets up to that year. The dataset has approximately 9900 DJ sets and 99100 songs. Figure 5.1 shows a DJ set from the dataset.
\\
\\
\\
\begin{figure}[H]
	\includegraphics[scale=0.45]{images/dataset}
	\centering
	\caption{Table showing a DJ Set in the dataset} 
\end{figure}


\section{Initial Suggestions}
As in Daniel Chows method, the initial suggestions part of the model is made using alternating least squares, a matrix factorisation algorithm that was made popular by the Netflix Prize award \citep{zhou_large-scale_2008}. As explained in Chapter 2, it provides a computational way of handling a dataset, and it splits the DJ song matrix to its latent factors, revealing trends in between songs and DJs.

A model is made from the dataset, then a list of songs gets inputted. A for loop is ran that will find similar items with the alternating least squares algorithm for each song. the number of recommended song suggestions for each song was set to 200, to assure a large amount of similar songs are used in the next part of the model.
\begin{figure}[H]{\noindent\ignorespaces}
	\includegraphics[scale=0.1]{images/application_app_flow}
	\centering
	\caption{A flow diagram of the application} 
\end{figure}

\section{Final Suggestions}
As of yet the code used is similar to Daniel Chow's, aside from iterating through many song inputs, instead of one \citep{chow_music_2020}. To add an extra layer of filtering, the Spotify API is used to further explore which songs are the most similar.

Using both the input songs, and initial suggestions, the Spotify API provides audio features for each song. The features used are shown below \citep{spotify_web_2023}:

\begin{enumerate}
	\item \textbf{acousticness} - A measure on how acoustic the song sounds (1 being definitely acoustic).
	\item \textbf{danceability} - Measurement on how suitable a track is for dancing based on tempo, rhythmic features and overall regularity. 1.0 is most danceable. 
	\item \textbf{energy} - A measurement of intensity and activity. Typically, energetic tracks feel fast, loud, and noisy. A heavy metal track would score high, but a Bach prelude would score low. Attributes include dynamic range, perceived loudness, timbre, onset rate, and general entropy will dictate this value. 1.0 is highly energetic.
	\item \textbf{instrumentalness} - Predicts whether a track contains no spoken vocals. Songs scoring 1 likely have no vocals. Values above 0.5 represent instrumental tracks.
	\item \textbf{key} - The key the track is in uses following notation 0 = C, 1 = C$\sharp$/D$\flat$, 2 = D. If key cannot be detected then value is set to -1.
	\item \textbf{loudness} - Loudness of a given track in dB
	\item \textbf{tempo} - The beats per minute of a given track.
	\item \textbf{time\_signature} - Time signature with the following notation, 3 = 3/4, 4= 4/4, 5 = 5/4.
	\item \textbf{speechiness} - Detects if a song has spoken word, audio books would score 1 and values ranging from 0.3-0.6 would be combination of music and speech.
	\item \textbf{valence} - A measurement of how positive a song is. With scores of 1 being extremely positive.
	
\end{enumerate}

These features provided great data to form vectors out of for each song, as they highlight attributes of a song which makes them unique. Attributes like tempo is especially helpful for improving the recommendations for DJ Sets, due to being the controlled variable when it comes to transition from one song to another.

The input contains many songs, so for calculating euclidean distance, a mean vector is created from all the input song vectors. The vectors are then scaled. This step is crucial because some attributes work with different ranges, examples being loudness uses dB range and acousticness is from 0-1.

Once these are found, euclidean distance between each initially suggested songs vector and the input average vector is calculated. As discussed in chapter 2, euclidean distance is a way of measuring the distance between two vectors. An equation is show below, allowing $x_{i}$ and $y_{i}$ to be two song vectors.

\begin{equation}
	d(x,y) = \sqrt{\sum _{i=1} ^{n}(x_{i} - y_{i})^{2}}
\end{equation}

For finding the vectors with the smallest distance between the mean input, this is a clear representation that a given song has similar Spotify features attributes, therefore would be a worthy suggestions based on the recommended songs.

\section{Example Findings}
To demonstrate the application in a non methodical way. A variety of DJ Set tracklists corresponding to different genres were used as inputs for the application.  Three different DJ sets were used:

\begin{itemize}
	\item \textbf{Rory Bowens, PLO Man, C3D-E, Brian Not Brian - The Slip, NTS Radio: } 
	\begin{itemize}
		\item \textbf{Genre:} New wave \& House
		\item \textbf{Tracks available:} 3/21
	\end{itemize}
	\item \textbf{Moxie, Shanti Celeste, Chris Farrell - NTS Radio: } 
	\begin{itemize}
		\item \textbf{Genre:} Tech-House
		\item \textbf{Tracks available:} 10/28
	\end{itemize}
	\item \textbf{Chimpo b2b L U C Y @ Keep Hush Live: Sherelle Presents:}
	\begin{itemize}
		\item \textbf{Genre:} Jungle \& Breaks
		\item \textbf{Tracks available:} 8/21
	\end{itemize}
	
\end{itemize}


When going through the recommended songs, the BPMs of the output songs were observed to see if they were around the same range as the input songs. The songs were listened to see if stylistically they were similar to the input songs, and the source of the recommended songs were also observed.

The Slip set performed okay. Expectations were low due to the small amount of tracks available. Two of them had a House style to them and one was Indie. The output songs were all House songs and none matched stylistically to the one indie song. However, all the BPM's were in a similar range around 110-120 BPM. The Moxie set performed very well. The input songs were all Tech-House songs and the output songs were that as well with a couple of more disco inspired tunes. BPM's were all very similar as well. The Keep Hush set also gave good results. The BPM ranges were similar to the input songs and stylistically matched the input songs well. 

All the suggested tunes did not come from an eclectic mix of DJ Sets or DJs, in each batch of suggestions a lot of songs were either from the same mix or the same DJ. This is most likely due to the sparsity of the dataset, which is value of 99.98\% is not ideal for a training set. 

The intent of displaying this is more to show the application in action rather than dissecting the quality of its performance which is analysed in the Experiment chapter. A table of the input and recommended songs for the Moxie DJ set is shown below.
\\
\\
\\
\begin{center}
	\begin{tabular}{ |c|} 
		\hline
		\textbf{Moxie, Shanti Celeste, Chris Farrell - NTS Radio}\\ 
		\hline \textbf{Input Songs}\\ 
		\hline Escape Earth \textit{Gravity Well} \\ 
		\hline Priori \textit{6thematic }\\
		\hline Carter Bros. \textit{Run - Monty's Bonus Beats}\\
		\hline Tyler Dancer \textit{Karmán Line}\\ 
		\hline Anunaku \textit{Forgotton Tales}\\
		\hline Piezo \textit{Tinned}\\
		\hline KMA Production \textit{Cape Fear}\\
		\hline Parris \textit{Dusty Glass Bubbles}\\
		\hline Tornado Wallace \textit{Open Door - Born Inna Tent Mix}\\
		\hline Luke Slater \textit{I Want You Too}\\
		\hline \textbf{Recommended Songs}\\ 
		\hline Black Booby \textit{Fill My Cup}\\
		\hline Omar S, John FM \textit{Heard'chew Single}\\
		\hline Hammer \textit{Manaka}\\
		\hline Awanto 3 \textit{Pregnant}\\
		\hline Steve Murphy \textit{Next Saturday - Club Mix}\\
		\hline Chasindub \textit{Still Here}\\
		\hline Ian Pooley \textit{Swing Mode}\\
		\hline Hashin \textit{Al-Naafyish (The Soul) - The "It's" About "Time" Remix}\\
		\hline Synkro \textit{Look at Yourself}\\
		\hline Kapote \textit{Uhh Baby - Brame \& Hamo Remix}\\
		\hline
	\end{tabular}
\end{center}


\section{Summary}
An overview of the proposed application was given. The first part described the alternating least squares method that is borrowed from Daniel Chow's model. Then the audio features found through the Spotify API was explained, and how it was used to form vectors. These vectors were then used to filter through the recommendations further. A non methodical demonstration was conducted showing pretty good recommendations.




% Chapter 6
% !TEX root = ../TechProject.tex

\graphicspath{{Chapter5/}}


\chapter{Experiment}


As explained in the hypothesis. a music recommendation system that uses MixesDB as a dataset was made. An experiment was ran to see if it garners appropriate results, and will help answer the following question:
\\
\\
\textit{Is the proposed method a suitable solution to automate recommendations of songs suitable for adding to a given DJ set?}.
\\

To evaluate the quality of a recommendation system, R-Precision is often used as a performance metric. 
\\

\section{R-Precision}

\begin{equation}
	R-Precision = \frac{|G\cap R_{1:|G|}|}{|G|}
\end{equation}
\\\\
R-precision is the amount of found tracks that were originally in the mix divided by the number of known missing tracks. This is shown in the equation in 6.1 allowing $G$ to be the set of missing songs, and $R$ to be the set of recommanded songs.

R-Precision was used in the 2018 Recsys Convention idea for testing submissions for the million playlist challenge dataset from Spotify \citep{aicrowd_aicrowd_2023}. At this convention, competing teams built recommendation system trained with the \textit{Spotify} dataset. To tests its quality, a challenge dataset, which contained many modified playlists, was used to test the models.

It is important to note the considerable difference in the number of DJ sets/playlists between the MixesDB and the Spotify dataset. The MixesDB dataset contains over 9800 DJ sets, while the Spotify dataset contains 1,000,000 playlists. The differences between playlists and DJ sets is worth mentioning as well. A playlist being a collection of songs, regardless of cohesion or song count, while a DJ set is typically a seamless mix of songs that can vary in length from 8 hours to no less than 20 minutes.
\\
\\
The top 10 scoring models from the convention have R-Precision values ranging from 0.21 to 0.22. Given the differences in dataset size and task, aiming for an average R-Precision value of 0.171, slightly lower than the top-performing models, would be appropriate. Achieving this value would rank the model within the top 30 models trained by members of the public on the given Spotify dataset \citep{aicrowd_aicrowd_2023}. This approach allows for a fair comparison between the different systems and considers the unique characteristics of the dataset and task at hand. 

As the dataset utilized with the proposed model was obtained from MixesDB, a customized approach for the preparation of the evaluation set was conducted.

\section{Preparing Evaluation Set}
During the evaluation phase of the million playlist challenge, the quality of a given applications trained on this dataset was assessed through a range of methods that involved analysing both missing and recommended songs. In order to effectively compare different applications, it was essential to create a standardised evaluation set that incorporated aspects of the Spotify data set, to show if a submitted applications could recommend songs that were featured in the original playlist.

\begin{figure}[H]
	\includegraphics[scale=0.5]{images/spotify_challenge_set}
	\centering
	\caption{Screenshot of spotify challenge set. \citep{aicrowd_aicrowd_2023}} 
\end{figure}

The Spotify challenge set consisted of 10,000 playlists with varying numbers of input songs, ranging from 0 to 100 tracks. For each of these playlists, Spotify requested 500 track recommendations from the participating teams \citep{aicrowd_aicrowd_2023}. However, due to the smaller scale of the dataset and difference between playlists and DJ sets under consideration, various aspects of the  challenge set used on the proposed model had to be downscaled. For instance, the range of input songs was considerably narrower, as DJ sets typically have a minimum length of 10 tracks, compared to playlists that can have varying lengths.
\\
\\
To ensure all songs in the evaluation set could be found, each song had to have a total play count of at least 5. This means each song had to be played a total of 5 times by other DJ's within the training set. This value was chosen to ensure that each song in the set would likely be found in the training set. Furthermore, the input and missing songs were split in the following way, with 20\% of the songs allocated as missing songs, and the remaining 80\% as input songs. This was done to ensure enough songs were inputted. The output value for the application was also changed from 10 to 100 to better match the output for the evaluation given for the \textit{Spotify} dataset.

The making of the evaluation set was coded in \textit{Python} using the \textit{pandas} library and \textit{pytest} was used for unit-testing.
\\
\\
\begin{figure}[H]
	\includegraphics[scale=0.1]{images/evaluation_set_app_flow}
	\centering
	\caption{Application flow of the creation of the evaluation set} 
\end{figure}

\section{Experiment Variables}

As mentioned in chapter 5, the proposed application utilises the alternating least squares algorithm to generate an initial set of suggestions. This algorithm is based on matrix factorization, where the latent factors are derived from the training dataset to identify similar songs. The selection of an appropriate value for latent factors is crucial in ensuring the optimal performance of the algorithm. A low latent factor value may yield random results , while a value that is too high may lead to over-fitting \citep{zhou_large-scale_2008}. To identify the ideal value for latent factors, it is necessary to conduct an  investigation where the latent factors will be the independent variable, and the average R-precision value will be the dependent variable. 

Once the best R-precision value is obtained, an examination on how changes to the size of the input DJ sets, impact the R-Precision value. From there, the R-precision value is then compared to the highest performing recommendation systems that use the Spotify Dataset, to see if the DJ set focused recommender can compare to industry standard models.

The experiment was ran on a 16GB 2020 MacBookPro.

\section{Summary}
An overview of the Spotify challenge set was given. The changes that had to be made to create a challenge set out of the MixesDB dataset was explained. It was  shown how the evaluation set was made. 

The description on the test plan was shown. The initial independent variable was latent factors, and then the input DJ set size. The experiment will be concluded with a comparision in R-Precision scores with top performing models trained on the Spotify dataset.



% Chapter 8
% !TEX root = ../TechProject.tex

\graphicspath{{Chapter7/}}

\chapter{Further Work}

A recomendation system that pulls from a dataset of exclusively DJ sets was proposed to see whether it gathers more suitable solutions compared to the current industry standard models available. 

The Spotify features play a big role on furthering down the similar songs to limit ones that have the most similar attributes to the input songs, allowing them to be easily incorporated into a DJ set. This proved helpful on cherry picking which were the best songs, however it did limit the scope of the application a fair bit. 

Despite Apple Music, Spotify, and the other leading subscription based streaming platforms having a huge library, there are a lot of songs, especially ones surrounding DJ sets that cannot be found on these platforms. Reasons for this is the elusive nature surrounding sub cultures that incorporates DJing in a significant way. It's very common for some artist to only release content on only physical formats like Vinyl. This is motivated by a fear of opening up there culture to a majority subset, that may risk of tainting a pure experience created from a select number of people. Another reason for this is the cost of sample clearance. More so than other genres (aside from Rap), sampling plays a huge role in dance music. Early instances of DJ-set orientated genres like house and techno were build on incorporating drum machines into old disco and soul records \citep{reynolds_energy_2013}. Favourable stylistic features aside, this creates a problem where the shareholders of the given samples are lawfully given the opportunity to sue a given artists a great sum of money for using there given work without permission. This means a huge amount of DJ set centric genres, pre the Digital Age has had issues being made commercially available, a highly reported instance of this is De La Souls 30 year long quest on gaining back there highly regarded musical output from the late 80's and 90's \citep{saunders_soul_2023}.

This inadvertently hinders the proposed recommendation system because it can only input and suggest songs that are also available on Spotify. Future work can use this website to scrape data from both Spotify and YouTube, which contains a huge amount of fan uploaded songs that aren't available on standard streaming platforms. Scraping data from services like Band-Camp would also help to populate the dataset further. Combining the software used to examine the Spotify features on the given audio from YouTube and other platforms would be beneficial for it to further mirror the proposed system, however doing this would certainly require a huge amount of processing power, which the 2020 16GB MacBook Pro the application was made in would not be able to handle.

A potential option that could've been implemented is for the user to be able to control the recommendation process more than just choosing the input songs. It's believable that there is a scenario where a user would appreciate the initial pool of recommendation just as much as the refined selection.

Despite being objectively outdated in a world where deep learning is the zeitgeist, using collaborative filtering as the basis of the model felt appropriate due to its continued usage in modern systems. The purpose of the experiment was more to see whether a dataset of this context could provide better recommendation than systems that use datasets from users who on average listen to music in a much  more casual manner. All this being said, adapting the model to use deep learning with this dataset would've been a great opportunity to see whether an improved model would show more desirable recommendation.

With mixesdb.com growing in popularity each year, the continued growth of internet radio, and the growth of DJ mix companies like Boiler Room, one hopes to see further exploration in using this dataset to cater to other DJ related tasks as it continues to be added with better archiving. With a sparsity of 99.98\% it does make sense why this would underperform on the given experiment, and hopefully with a more thorough dataset another exploration into seeing whether a dataset this specific for certain genres would be more beneficial compared to models trained on more casual listeners. 

There was opportunities where the Spotify features euclidean distance filtering could've been implemented better. An example being was adding weights to certain attributes. An example being prioritizing songs at a certain BPM or key. The weighting for key could've been explored to an interesting length, by prioritising the same key but also the relative major/minor or subdominant or dominant keys.

Another adjustment that could've been made to the euclidean distance part of the code is only finding an average input vector with songs  in the input that have similar Spotify features. A problem with having one outlier song is that it could disproportion the input average and spoil the quality of the songs recommended. This problem could be resolved by ignoring outlier songs, find euclidean distances with the outlier songs and pool from the outliers suggestions sparingly in the final batch.

For testing there were several ways the application could've been assessed. 

\subsection{Recommended Song Clicks}
When you add songs to a Spotify playlists, 10 songs are recommended based on what's in the playlist. The user can refresh the list and it gives out 10 more similar songs. Recommended Songs clicks is the amount of refreshes before a missing track shows up. Here is the following equation.

\begin{equation}
	Clicks= \lfloor \frac {argmin_{i}  \{ R_{i} : R_{i} \in G | \} -1} { 10 } \rfloor
\end{equation}


\subsection{Spotify Features}
Another form of testing will be to use the Spotify features in conjunction with the Spotify playlist. We will input 10 songs into both my application and a Spotify playlists. Get the Spotify features of both outputs and compare which ones are the most suitable.

\subsection{Spectrogram Analysis}
With the dataset being linked to Spotify, a 30 second clip for each song is accessible. Assess the spectrograms of the input songs compared to the output songs would've given further insight if the songs suggested were appropriate.
 
\subsection{Listening Tests}
Gathering members of the public and getting them to listen to snippets of a given DJ set and compare the applications suggestions and industry standard suggestions would've given further insight on the subjective quality of the application.



% note that \Blindocument has 5 numbered levels, despite setting secnumdepth above. I (and many style guides) would suggest using no more than 3 numbered levels (incl. the chapter), with the option of a fourth unnumbered level.

%% Appendices
%% appendices work in an identical way to chapters (in terms of syntax), but are labelled alphabetically rather than numerically
\appendix
% Replace "Chapter" with "Appendix"
\addtocontents{toc}{\protect\renewcommand{\protect\cftchappresnum}{\appendixname~}}
\addtocontents{toc}{\protect\addtolength{\cftchapnumwidth}{1em}}
%\input{Appendix1}
%\input{Appendix2}

% Last I checked, the Appendix and Titlesec package do not get on with each other. So use "input" for appendix chapters instead. If you still want to use \includeonly, put the \input commands in another file and \include that.

%%%%%%%%%%%%%%%%% BACK MATTER %%%%%%%%%%%%%%%%%
\backmatter

% header
\renewcommand{\chaptermark}[1]{\markboth{#1}{}}

% !TEX root = ../TechProject.tex

\graphicspath{{BackMatter/}}

\phantomsection

%% add references here
\bibliographystyle{iosrnew.bst}
%\bibliography{TechProjbib.bib} your bib file name

\bibliography{BackgroundReading.bib}
\end{document}
