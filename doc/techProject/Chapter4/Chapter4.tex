% !TEX root = ../TechProject.tex

\graphicspath{{Chapter4/}}

\chapter{Hypothesis}

\section{Literature Review Conclusions}

The literature review set out to answer the two questions:
 
 \begin{enumerate}
 	\item \textit{What are effective methods for the automated recommendation of songs suitable for adding to a given DJ set?}
 	
 	\item \textit{Can the proposed method be further developed to solve other DJ tasks?}
	\end{enumerate}
 
 It was shown that the current machine learning advances in separation and classification could aid a DJ greatly, but despite great advances in recommendation systems, algorithmic ways of gathering song recommendations do not handle the cultural significances of genre and style.

With this being the case, modern day recommendation systems don't usually cater to the traits found in both professional and hobbiest DJs. Preparing a DJ set requires a sizeable amount of music \citep{allen_djs_2021}. Based on the research ran by Anderson, these group of people would certainly fall in the category of diverse \citep{anderson_algorithmic_2020}.

The necessity to reach a certain quantity of songs to DJ, makes for a laborious passed time of collecting or "digging" for songs. Digging is a pre digital era term, to go find music in specific record stores or from esoteric places, with the higher likelihood of picking up something that could be emotionally resonant \citep{allen_djs_2021}.

Recommendation systems could be seen as a replacement for digging. A given application should knows if a user want esoteric music. But as mentioned in the Diversity Issue sub-section, finding these sorts of recommendations relies heavily on the cultural context surrounding a genre and with it gives less desirable suggestions compared to manually finding music. 

This problem is somewhat resolved with the popularity behind sites like "bandcamp", that prides itself with regularly updated blog posts, and the option to examine other people collections. Allowing the user to openly "dig" more so than other music based platforms \citep{bandcamp_about_2023}.  These sites can be a great resource to find music in a digging manner, but having an algorithm do it would be more time efficient. 

Successfully training a model from this type of dataset could then be used to assist other DJ related tasks, an example of parameter values was given.

Aside from Daniel Chow's model, a recommendation system that attempts to mimic suggestions found from digging is yet to happen \citep{chow_music_2020}. However, ways of adapting modern day recommendation systems can mimic the effect.

\section{Algorithmic ways of obtaining recommendations from a DJ set}
\begin{enumerate}
	\item \textbf{Spotify }- Not many can be found, but labels like !K7 publish DJ sets commercially \citep{k7_about_2023}.  Find a DJ set on Spotify, add that to a playlist, and look for the"
	Recommended based on the playlist" . The main problem with this is that DJ sets on Spotify are few and far between. One can only input a mix one’s heard from the radio or a recorded live performance by finding the tracklist and adding the songs into a playlist. The system also finds what other users are listening to, often giving song recommendations that are usually quite popular and well-known. Spotify has a vast library of music, but a sizeable chunk of music played on Spotify are not found on DJ sets.
	
	\item \textbf{SoundCloud }- Many DJ sets are found here, and using the "radio" function, recommendations can be found. However, it prioritises other DJ sets on its "radio " when getting recommended songs is more desirable in this context. The recommendation system is also only limited to what is available on SoundCloud.
	
\end{enumerate}

\section{Hypothesis proposal}
	
The hypothesis is that the application will give song recommendations that are both stylistically suitable but aren’t necessarily popular. Its common for a DJ, having a diverse taste, to go into
great lengths to find obscure and unknown songs. Therefore, not prioritising popularity is a
desirable trait in a recommendation system for a DJ’s music consumption and knowledge, usually
being more than the average consumer. Its a common trait in most music recommendation
systems to prioritise popularity. Building a recommendation system from a dataset of DJ sets
will hopefully give these types of suggestions.	

With DJ sets being the intended input of this model, one hopes that its accuracy is comparable with top performing recommendation system models.

\section{Summary}
The literature review was summarised and the gaps in research was stated. An overview of obtaining recommendations for a DJ set was given. A hypothesis given  hopes the accuracy of the model would be comparable to top performing model, when DJ sets are used as inputs. 



% note that \Blindocument has 5 numbered levels, despite setting secnumdepth above. I (and many style guides) would suggest using no more than 3 numbered levels (incl. the chapter), with the option of a fourth unnumbered level.