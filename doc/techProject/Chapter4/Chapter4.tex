% !TEX root = ../TechProject.tex

\graphicspath{{Chapter4/}}

\chapter{Hypothesis}

\section{Literature Review Conclusions}

The literature review set out to answer the two questions:

\begin{enumerate}
	\item \textit{What are effective methods for the automated recommendation of songs suitable for adding to a given DJ set?}
	
	\item \textit{Can a DJ-set centric dataset be used to solve other DJ tasks?}
\end{enumerate}

It was shown that machine learning advances in separation and classification could aid a DJ greatly. However, despite significant advances in recommendation systems, algorithmic ways of gathering song recommendations do not handle the cultural significance of genre and style.

With this being the case, modern-day recommendation systems do not usually cater to the traits found in both professional and hobbyist DJs. Preparing a DJ set requires a sizeable amount of music \citep{allen_djs_2021}. Based on the research run by Anderson, DJs would undoubtedly fall in the category of diverse \citep{anderson_algorithmic_2020}.

Reaching a certain quantity of songs to DJ requires collecting or "digging" for songs. Digging is a pre-digital era term to find music in specific record stores or from esoteric places, with the higher likelihood of picking up something that could be emotionally resonant \citep{allen_djs_2021}.

Recommendation systems could be seen as a replacement for digging. A given application should know if a user wants esoteric music. However, as mentioned in the Diversity Issue sub-section, finding these sorts of recommendations relies heavily on the cultural context surrounding a genre and gives less desirable suggestions than manually finding music. 

This problem is somewhat resolved with the popularity behind sites like Bandcamp, which prides itself on regularly updated blog posts, and the option to examine other people's collections. These features encourage users to "dig" more than other music-based platforms \citep{bandcamp_about_2023}.  These sites can be a great resource to find music in a digging manner, but having an algorithm do it would be more time efficient. 

Successfully training a model from a DJ-centric dataset could be used to suggest quality songs and potentially assist other DJ-related tasks algorithmically.

Aside from Daniel Chow's model, a recommendation system that attempts to mimic suggestions found from digging is yet to happen \citep{chow_music_2020}. Adapting modern-day recommendation systems to this task gives lacklustre results.

\section{Algorithmic ways of obtaining recommendations from a DJ set}
\begin{enumerate}
	\item \textbf{Spotify }- Find a DJ set on Spotify, add that to a playlist, and look for "Recommended based on the playlist". The main problem with this is that DJ sets on Spotify are few and far between. Not many can be found, but labels like !K7 publish DJ sets commercially \citep{k7_about_2023}. The system also finds what other users are listening to, often giving song recommendations that are usually quite popular and well-known.
	
	\item \textbf{SoundCloud }- Many DJ sets are found here, and recommendations can be found using the "radio" function. However, it prioritises other DJ sets on its "radio" rather than songs. The recommendation system is also only limited to what is available on SoundCloud.
	
\end{enumerate}

\section{Hypothesis proposal}

The hypothesis is that the application will give song recommendations that are both stylistically suitable but not necessarily popular. It is common for a DJ, having a diverse taste to go to
great lengths to find obscure and unknown songs. Therefore, not prioritising popularity is a
desirable trait in a recommendation system for a DJ's music consumption and knowledge, usually
being more than the average consumer. It is a common trait in most music recommendation
systems to prioritise popularity. Building a recommendation system from a dataset of DJ sets
will hopefully give these types of suggestions.	

With DJ sets being the intended input of this model, its accuracy should be comparable with top-performing recommendation system models.

\section{Summary}
The literature review was summarised, and the research gaps was stated. Finally, an overview of obtaining recommendations for a DJ set was given. A hypothesis expects the model's accuracy to be comparable to the top-performing model when DJ sets are used as inputs. 



% note that \Blindocument has 5 numbered levels, despite setting secnumdepth above. I (and many style guides) would suggest using no more than 3 numbered levels (incl. the chapter), with the option of a fourth unnumbered level.