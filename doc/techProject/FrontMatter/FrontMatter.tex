% !TEX root = ../TechProject.tex

\graphicspath{{FrontMatter/}}

%%%%%%%%%%%%%%%%% TITLE PAGE %%%%%%%%%%%%%%%%%
\begin{titlepage}
\thispagestyle{empty}
\hfill\includegraphics[width=50mm]{images/UniS_Logo.pdf}
{\sf
\centering
\null\vfil\vfil
	{
		\huge\LongTitle\\
	}
\vfil\vfil\vfil
	{
		\Large\Me\\
	}
\vfil\vfil\vfil
	{
		\Large{}BMus/BSc Music \& Sound Recording (Tonmeister)\\
		Technical Project\\
	}
\vfil\vfil
	{
		\Large{}Institute of Sound Recording\\
		University of Surrey\\
	}
\vfil
	{
		\Large\MyMonth\today~\MyYear\today\\}
	}

	{
		\includegraphics[width=40mm]{images/University_of_Surrey_coat_of_arms.pdf}
		\vspace{-.4in}
	}
\end{titlepage}

\setlength{\parskip}{1ex plus 0.2ex minus 0.2ex} % set paragraph spacing
\onehalfspacing

%%%%%%%%%%%%%%%%% ABSTRACT %%%%%%%%%%%%%%%%%

\pdfbookmark[0]{Abstract}{abstract}

\chapter*{Abstract}
\thispagestyle{empty}
\setcounter{page}{2}

Music recommendation systems are used to filter down the vast amount of commercially available music to a digestible and personable amount. Recommendation systems have seen significant improvement recently due to advances in deep learning. However, listeners with diverse musical taste still do not use recommendation systems. This is evident with the continued popularity of internet radio and the desire for "curation" found in streaming services. DJs are in this diverse category, and considering the amount of investigation needed to prepare a DJ set, an algorithmic way of preparing a DJ set is desirable. With the vast amount of DJ sets available online, training a recommendation system on DJ sets could resolve this issue and other DJ-related tasks.

In this project, a novel DJ-set recommendation model is proposed. The model is trained on DJ sets, available on MixesDB, an online database for DJ sets. It makes use of the matrix factorisation algorithm, alternating least squares. The audio features used for this model were obtained from Spotify API and then leveraged to improve suggestions further.

An experiment was also conducted that tests the R-Precision of the application, an evaluation metric used for the million playlist Spotify challenge. An evaluation set was made, and the parameters for obtaining the highest average R-precision were investigated. The proposed model operated successfully; however, compared to the top-performing models, the model underperformed significantly. It's thought that the sparsity of the dataset is the main contributor to its results.

%%%%%%%%%%%%%%%%% ACTUAL FRONT MATTER %%%%%%%%%%%%%%%%%
\frontmatter

\singlespacing

\setlength{\parskip}{0ex} % set paragraph spacing

\clearpage
\pdfbookmark[0]{Contents}{Contents}
\phantomsection
\tableofcontents
\clearpage
\phantomsection
\addcontentsline{toc}{chapter}{List of Figures}
\listoffigures
\clearpage
\phantomsection
\addcontentsline{toc}{chapter}{Bibliography}


\setlength{\parskip}{1ex plus 0.2ex minus 0.2ex} % set paragraph spacing

\onehalfspacing

\chapter{Acknowledgements}

Thanks to Christos for the encouragement and advice whilst doing this. Thank you to my parents and my friends for always being loving and compassionate. Thank you to Malibu for sound-tracking a huge majority of the time put into this.