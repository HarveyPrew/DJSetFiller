% !TEX root = ../TechProject.tex

\graphicspath{{Chapter7/}}

\chapter{Further Work}

I proposed a reccomendation system that pulls from a dataset of exclusively DJ sets, to see whether it gathers more suitable solutions compared to the current baseline models available. 

The spotify features play a big role on furthering down the similar songs to limit ones that have the most similar attributes to the input songs, allowing them to be easily incorperated into a DJ set. This proved helpful on cherry picking which were the best songs, however it did limit the scope of the application a fair bit. 

Despite Apple Music, Spotify, and the other leading subscription based streaming platforms having a huge library, there are a lot of songs, especially ones surrounding DJ sets that cannot be found on these platforms. Reasons for this is the elusive nature surrounding sub cultures that incorpoerate DJing in a significant way. It's very common for some artist to only release content on physical formats like Vinyl, of a fear of opening up there culture to a majority subset that may risk of tainting a pure experience created from a select number of people. Another reason for this is the cost of sample clearence. More so than other genres (aside from Rap), sampling plays a huge role in dance music. Early instances of DJ-set orientated genres like house and techno were build on incorperating drum machines into old disco and soul records. Favourable stylistic features aside, this creates a problem where the shareholders of the given samples are lawfully given the opportunity to sue a given artists a great sum of money for using there given work without permission. This means a huge amount of DJ set centric genres, pre the Digital Age has had issues being made comercially available, a highly reported instance of this is De La Souls 30 year long quest on gaining back there highly regarded musical output from the late 80s and 90s \citep{saunders_soul_2023}.

This inadvertently hinders the proposed recommendation system because it can only input and suggest songs that are also available on spotify. Future work can use this website to scrape data from both spotify and YouTube, which contains a huge amount of fan uploaded songs that arent available on standard streaming platforms. Combining the software used to examine the Spotify features on the given audio from YouTube would be helpful for it to further mirror the proposed system, however doing this would certainly require a huge amount of processing power, which the 2020 16GB MacBookPro the application was made in would not be able to handle.

A potential that would be very easy to implement is the user to be able to control the recommendation process more than just choosing the input songs. I believe there would be a scenario where a user would appreciate the initial pool of recommendation just as much as one that much the stylistic features of the input songs more so.

Despite being objectively outdated, I found using collaborative filtering as the basis of my model felt appropriate due to its continued usage in modern systems. The purpose of the experiment was more to see whether a dataset of this context could provide better recommendation than systems that use datasets from users who on average listen to music in a much  more casual manner. All this being said, adapting the model to use deep learning with this dataset would've been a great opportunity to see whether an improved model would show more desirable recommendation.

With mixesdb.com growing in popularity each year, the continued growth of internet radio, and the growth of DJ mix companies like Boiler Room, I hope to see further exploration in using this dataset to cater to other DJ related tasks as it continues to be added with better archiving. 

This only appears on a handful of mixes, but a time code along with the setlist could provide the foundations for a very helpful DJ tool. The act of a user inputting a song, and be able to hear instances of that song getting mixed in and out of could help speed up the learning process when they immediately have an audio reference of potentially effective mixing. With this being a forever growing hobby and career for the select who are lucky enough, the fact a system that works in this way is an oddity in the big data era we live in. 

For testing there were a couple of ways I could've tested the application. 

\subsection{Recommended Song Clicks}
When you add songs to a Spotify playlists, 10 songs are recommended based on what's in the playlist. The user can refresh the list and it gives out 10 more similar songs. Recommended Songs clicks is the amount of refreshes before a missing track shows up. Here is the following equation.

\begin{equation}
	Clicks= \lfloor \frac {argmin_{i}  \{ R_{i} : R_{i} \in G | \} -1} { 10 } \rfloor
\end{equation}


\subsection{Spotify Features}
Another form of testing will be to use the spotify features in conjunction with the spotify playlist. We will input 10 songs into both my application and a spotify playlists. Get the spotify features of both outputs and compare which ones are the most suitable.
 


% note that \Blindocument has 5 numbered levels, despite setting secnumdepth above. I (and many style guides) would suggest using no more than 3 numbered levels (incl. the chapter), with the option of a fourth unnumbered level.