% !TEX root = ../TechProject.tex

\graphicspath{{Chapter7/}}

\chapter{Further Work}

A recomendation system that pulls from a dataset of exclusively DJ sets was proposed to see whether it gathers more suitable solutions compared to the current industry standard models available. 

The Spotify features play a big role on furthering down the similar songs to limit ones that have the most similar attributes to the input songs, allowing them to be easily incorporated into a DJ set. This proved helpful on cherry picking which were the best songs, however it did limit the scope of the application a fair bit. 

Despite Apple Music, Spotify, and the other leading subscription based streaming platforms having a huge library, there are a lot of songs, especially ones surrounding DJ sets that cannot be found on these platforms. Reasons for this is the elusive nature surrounding sub cultures that incorporates DJing in a significant way. It's very common for some artist to only release content on only physical formats like Vinyl. This is motivated by a fear of opening up there culture to a majority subset, that may risk of tainting a pure experience created from a select number of people. Another reason for this is the cost of sample clearance. More so than other genres (aside from Rap), sampling plays a huge role in dance music. Early instances of DJ-set orientated genres like house and techno were build on incorporating drum machines into old disco and soul records \citep{reynolds_energy_2013}. Favourable stylistic features aside, this creates a problem where the shareholders of the given samples are lawfully given the opportunity to sue a given artists a great sum of money for using there given work without permission. This means a huge amount of DJ set centric genres, pre the Digital Age has had issues being made commercially available, a highly reported instance of this is De La Souls 30 year long quest on gaining back there highly regarded musical output from the late 80's and 90's \citep{saunders_soul_2023}.

This inadvertently hinders the proposed recommendation system because it can only input and suggest songs that are also available on Spotify. Future work can use this website to scrape data from both Spotify and YouTube, which contains a huge amount of fan uploaded songs that aren't available on standard streaming platforms. Scraping data from services like Band-Camp would also help to populate the dataset further. Combining the software used to examine the Spotify features on the given audio from YouTube and other platforms would be beneficial for it to further mirror the proposed system, however doing this would certainly require a huge amount of processing power, which the 2020 16GB MacBook Pro the application was made in would not be able to handle.

A potential option that could've been implemented is for the user to be able to control the recommendation process more than just choosing the input songs. It's believable that there is a scenario where a user would appreciate the initial pool of recommendation just as much as the refined selection.

Despite being objectively outdated in a world where deep learning is the zeitgeist, using collaborative filtering as the basis of the model felt appropriate due to its continued usage in modern systems. The purpose of the experiment was more to see whether a dataset of this context could provide better recommendation than systems that use datasets from users who on average listen to music in a much  more casual manner. All this being said, adapting the model to use deep learning with this dataset would've been a great opportunity to see whether an improved model would show more desirable recommendation.

With mixesdb.com growing in popularity each year, the continued growth of internet radio, and the growth of DJ mix companies like Boiler Room, one hopes to see further exploration in using this dataset to cater to other DJ related tasks as it continues to be added with better archiving. With a sparsity of 99.98\% it does make sense why this would underperform on the given experiment, and hopefully with a more thorough dataset another exploration into seeing whether a dataset this specific for certain genres would be more beneficial compared to models trained on more casual listeners. 

There was opportunities where the Spotify features euclidean distance filtering could've been implemented better. An example being was adding weights to certain attributes. An example being prioritizing songs at a certain BPM or key. The weighting for key could've been explored to an interesting length, by prioritising the same key but also the relative major/minor or subdominant or dominant keys.

Another adjustment that could've been made to the euclidean distance part of the code is only finding an average input vector with songs  in the input that have similar Spotify features. A problem with having one outlier song is that it could disproportion the input average and spoil the quality of the songs recommended. This problem could be resolved by ignoring outlier songs, find euclidean distances with the outlier songs and pool from the outliers suggestions sparingly in the final batch.

For testing there were several ways the application could've been assessed. 

\subsection{Recommended Song Clicks}
When you add songs to a Spotify playlists, 10 songs are recommended based on what's in the playlist. The user can refresh the list and it gives out 10 more similar songs. Recommended Songs clicks is the amount of refreshes before a missing track shows up. Here is the following equation.

\begin{equation}
	Clicks= \lfloor \frac {argmin_{i}  \{ R_{i} : R_{i} \in G | \} -1} { 10 } \rfloor
\end{equation}


\subsection{Spotify Features}
Another form of testing will be to use the Spotify features in conjunction with the Spotify playlist. We will input 10 songs into both my application and a Spotify playlists. Get the Spotify features of both outputs and compare which ones are the most suitable.

\subsection{Spectrogram Analysis}
With the dataset being linked to Spotify, a 30 second clip for each song is accessible. Assess the spectrograms of the input songs compared to the output songs would've given further insight if the songs suggested were appropriate.
 
\subsection{Listening Tests}
Gathering members of the public and getting them to listen to snippets of a given DJ set and compare the applications suggestions and industry standard suggestions would've given further insight on the subjective quality of the application.



% note that \Blindocument has 5 numbered levels, despite setting secnumdepth above. I (and many style guides) would suggest using no more than 3 numbered levels (incl. the chapter), with the option of a fourth unnumbered level.