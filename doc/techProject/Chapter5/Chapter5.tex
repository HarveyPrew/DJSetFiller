% !TEX root = ../TechProject.tex

\graphicspath{{Chapter5/}}

\chapter{Experiment}

As explained in the hypothesis. I'm testing a music recommendation system that uses MixesDB as a dataset and see whether it garners appropriate results.

To effectively evaluate this I choose a handful of mixes that have songs that appear in multiple mixes.

I then remove these songs from the said DJ set, make them the input, then evaluate how appropriate the suggestions are.

The idea for the testing came from the million playlist dataset from Spotify \citep{aicrowd_aicrowd_2023}.  Where they both had the dataset (million untouched Spotify playlists) and the challenge dataset, which contained many modified playlists, showing how many missing songs are in each playlists.

I will borrow two methods of testing from this data set, R-Precision and Recommended Song Clicks. I will then do a final testing using the spotify features

\subsection{R-Precision}
The first mean of testing would be finding the R-precision. The R-precision is the amount of found tracks that were originally in the mix divided by the number of known missing tracks.

\begin{equation}
	R-Precision = \frac{|G\cap R_{1:|G|}|}{|G|}
\end{equation}


\subsection{Recommended Song Clicks}
When you add songs to a Spotify playlists, 10 songs are recommended based on what's in the playlist. The user can refresh the list and it gives out 10 more similar songs. Recommended Songs clicks is the amount of refreshes before a missing track shows up. Here is the following equation.

\begin{equation}
	Clicks= \lfloor \frac {argmin_{i}  \{ R_{i} : R_{i} \in G | \} -1} { 10 } \rfloor
\end{equation}


\subsection{Spotify Features}
Another form of testing will be to use the spotify features in conjunction with the spotify playlist. We will input 10 songs into both my application and a spotify playlists. Get the spotify features of both outputs and compare which ones are the most suitable.
